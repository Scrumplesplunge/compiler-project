In Occam, any of the \texttt{IF}, \texttt{SEQ}, or \texttt{PAR} constructs can
be \textit{replicated}. What this means depends upon the construct:
\lstinputlisting[language=occam]{code/if.occ}
Here, the \texttt{IF} statement on lines 5-7 is equivalent to the following:
\lstinputlisting[language=occam]{code/if2.occ}
As you can see, allowing replication on \texttt{IF} statements drastically
reduces the amount of code required in such cases, without complicating the
simple cases.

Applying replication to \texttt{SEQ} is roughly equivalent to a \texttt{for}
loop in C, while applying replication to \texttt{PAR} is like a \texttt{for}
loop which runs all of its iterations in parallel.

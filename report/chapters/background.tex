This project has a heavy focus on the programming language Occam, and the
Transputer microprocessor that the language was designed for. This section
explores the details relevant to the design of an Occam compiler, and of a
Transputer emulator.

\section{Formal Languages}

A key part of understanding any compiler is knowledge of some of the different
classes of formal language. The two that are most important to compilers are
\textit{\gls{reglang}s} and \textit{\gls{cflang}s}.

\subsection{Regular Languages}

Regular Languages come into compilers at the \textit{\gls{lexing}} stage, as most
languages can be tokenised in such a way that each token can be matched by a
\gls{reglang}. In this project, regular expressions are not explicitly used as
part of the implementation, but serve as a manner of summarising how the input
is split into tokens.

\subsection{Context-Free Languages}

A lot of programming languages are, at their heart, context-free. The advantage
of this is that the structure of \textit{\gls{cfg}s} makes them ideal for
compiling code: the individual rules can each be matched by code which compiles
that particular construct. The grammar rules for the language are defined as a
context-free grammar, and the compiler implements a top-down backtracking
technique in order to parse them.

\section{Occam}

Occam is a procedural concurrent programming language that was introduced in the
1980s. The language has a very recognisable syntax in which indentation is used
to denote blocks of related code, and communication between concurrent threads
is performed via message passing through CSP-like channels. For the purpose of
this project and for the sake of simplicity, I will be focusing on the first
version of Occam.

\subsection{Basic Syntax}

By today's standards, Occam has an unusual syntax:

\begin{lstlisting}[language=occam]
PROC foo(VALUE a, b, CHAN c, d) =
  PAR i = [0 FOR n]  -- Replicated parallel.
    SEQ              -- Explicit sequence.
      VAR x :        -- Declare variable x.
      bar(x, i)      -- Call procedure bar.
      IF             -- If statement.
        x < b        -- if (x < b) {
          c ! a      --   Output a to channel c.
        x >= b       -- } else if (x >= b) {
          d ! ANY    --   Output something (it doesn't matter what) to d.
:                    -- }

PROC bar(VAR x, VALUE i) =          -- Occam has no operator precedence:
  x := ((i * i) \ (i + i + i)) - i  -- It is necessary to bracket all except
:                                   -- sequences of some associative operator.

CHAN c, d :
VAR x, e, f :
PAR
  foo(-1, 1, c, d)   -- Run the process foo.
  SEQ i = [0 FOR n]  -- Loop n times:
    ALT              -- Alternative: perform one of the following:
      c ? x          -- Read a value x from c.
        e := e + x
      d ? ANY        -- Read something (it doesn't matter what) from d.
        f := f + 1
\end{lstlisting}

The above (meaningless) program demonstrates some of the details of the
language. Note that unlike many other languages, Occam requires sequences of
statements to be explicitly denoted by the keyword \texttt{SEQ}. In fact, the
textual syntax is remarkably similar to the abstract syntax that one might
expect to find in the compiler.

\section{The Transputer}

The Transputer was an advanced processor which presented several advanced
features aimed at facilitating concurrent programming. The processor supported
synchronous communication over each of its four external serial connections, as
well as time-sliced concurrent processes, all through microcoded instructions
that made it possible to do these things with only a handful of assembly.

More like a microprocessor than merely a CPU, the Transputer incorporated all
the circuitry required for it to function, and it was the intention of INMOS to
market the Transputer as a one-size-fits-all solution for both areas. The aim
was for each unit to cost no more than a few dollars, and for it to be used for
everything from basic device controllers to hosting operating systems.

\subsection{Instruction Set} \label{ops}

The Transputer had an unusually high clock speed for its time. This was mostly
possible due to it being a RISC architecture. The transputer supports 13
\textit{\gls{direct}s}, each 8 bits, in its instruction set. Each of these
has room for a 4-bit operand. If room for an operand larger than 4 bits is
required then one of the additional instructions \texttt{PFIX} (prefix) or
\texttt{NFIX} (negative prefix) can be used to prepend additional bits. In
addition to these instructions, there is also the \texttt{OPR} (operate)
instruction, which interprets its operand as one of the many
\textit{\gls{indirect}s} available, and executes it.

\subsection{Registers}

The transputer does not have random access registers like many modern RISC
architectures. Instead, it has a small collection of single-purpose registers
which are accessed via specific instructions, and three general-purpose
registers which are arranged in a stack. However, some instructions can clobber
the contents of this stack, so it is intended to be used only for passing
arguments to instructions. Instead, the programmer should store the results of
all computations into the \textit{\gls{workspace}} of the process. This is
similar to a conventional falling stack, but with the added detail that the
workspace uniquely identifies the process that it belongs to.

\subsection{Inter-process Communication}

As mentioned above, inter-process communication on the Transputer is achieved
via message passing. What makes this impressive is that to the programmer, there
is no difference (other than a memory address) between communicating between
processes on the same Transputer, and communicating between two different
Transputers. The instructions remain the same, and the difference is handled in
the microcode. If one process tries to send a message to another before the
second is ready for it, then the first will wait until the second is ready
before sending, and it will not proceed until the second has received the
message. In this way, the channels can be used to synchronise separate threads.

\subsection{Concurrency}

The Transputer provides a small selection of instructions which schedule new
processes to run alongside the current. Only one process is truly running at any
given time, but the processor time-slices between the scheduled processes so
that all may proceed. It does so when particular instructions are performed
which can cause the process to yield control to another. If a process performs a
blocking action (eg. waiting upon a channel) then it will be descheduled until
it can proceed. The Transputer has two different priority levels which processes
can be executed at: low-priority and high-priority. If there is any
high-priority process which is currently scheduled, then it is guaranteed to
make progress before any low-priority process.

The processes are arranged in two linked lists; one for each priority level.
When a process is scheduled, it appears in one of these linked lists. The front
and back pointers of each list are stored in registers. Each pointer consists of
the address of the workspace of some process, with the priority of the process
encoded in the least significant bit of the workspace address\footnote{Workspace
addresses must be aligned to word boundaries, so the least significant two bits
can be safely ignored when recovering the workspace address.}. The workspace of
a descheduled process contains additional information below the workspace
pointer, such as the \texttt{Iptr} (instruction pointer) of the descheduled
process.

\subsection{Parallelism}

Although it has built-in support for having multiple processes running at once,
the Transputer does not implement true parallelism. The multiprocessing is
achieved by rapid switching between the active processes, and so there is no
performance gain made by having two processes instead of one. Rather, the point
of the multiprocessing support is to help break down programs into more
manageable processes which can act concurrently.

However, this does not mean that the Transputer is incapable of true
parallelism: the microprocessor comes equipped with four external serial
connections over which the Transputer is able to perform synchronous channel
operations. From the perspective of the programmer, the manner by which this is
achieved is exactly the same as if the Transputer was performing communication
over an internal channel, which means that the code does not have to be compiled
with special knowledge of which channels are internal and which are external.

By combining several Transputers together, true parallelism is therefore quite
achievable.

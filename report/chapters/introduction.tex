Many programming languages support concurrency and parallelism, but often this
functionality was added as an afterthought in libraries or as extensions. Very
few languages make multithreading simple, and fewer still do so without risking
data-races or other multithreading catastrophes.

\textit{Communicating Sequential Processes}, or \textit{CSP}, is a formal
language presented in 1984 by Brookes, Hoare, and Roscoe. The language is an
algebra for describing concurrent processes, and solves the problem of
thread-safe communication by synchronising on \textit{events}. In
implementation, these events correspond to synchronous channels that allow two
processes to communicate with each other by explicitly passing data between
themselves.

Not many languages choose to implement this approach to multithreading, instead
opting for primitives such as \textit{mutexes} and \textit{condition variables}.
Occam is a programming language which was designed to use these synchronous
channels instead, and this report will cover the implementation of a compiler
for Occam that targets a modern computer. Further, the report will show how the
language can be adapted so that the programs may run on a collection of machines
rather than just one, allowing for massive parallelism.

The report will start by laying out the objectives and the scope of this
project, and will then cover the design and implementation of an Occam compiler
in detail, before going on to cover the construction of a distributed back-end
that allows the language to truly take advantage of any available parallelism.

This project has been a rigorous and first-hand experience in not only writing
a compiler, but producing an efficient distributed runtime system for
simple multiprocessing.

\section{The Results}

The final state of the project has a collection of six applications:

\begin{itemize}
  \item \texttt{occ} - The Occam compiler.
  \item \texttt{as} - The assembler.
  \item \texttt{das} - The disassembler.
  \item \texttt{vm} - The virtual machine binary for fast single-threaded execution.
  \item \texttt{master} - The process master binary for distributed execution.
  \item \texttt{worker} - The worker binary for distributed execution.
\end{itemize}

Together, these form a fully useable collection of utilities. However, many of
the features mentioned in \ref{beyond-scope} would be highly desirable if these
were to be put into regular use. In summary, I think that I have achieved what
I set out to achieve: build a compiler which generates distributed programs.

\section{Continuation}

There are literally hundreds of things that could be done to extend this
project. These range from relatively benign things such as fleshing out the
semantic analysis, to more major changes such as adding support for floating
point values or compound datatypes. However, there are some particularly
interesting ideas which I intended to try if I found the time:

\subsection{Remote Memory Access}

When I implemented the instance spawning code and decided to make the workspace
of the new process start at the same address as the parent workspace at the time
of spawning the child, I had in mind the idea of potentially allowing child
instances to read and write the variables defined in the workspaces of their
ancestors.

In order to implement this efficiently, reads to external addresses could fetch
chunks in a similar fashion to fetching cache lines, and writes could be locally
cached and only be flushed when the instance exited. Due to the restrictive
nature of Occam's memory model, this would not be detectably different from
performing the reads and writes directly, but it would be much faster, and would
effectively make the \texttt{DIST PAR} construct a drop-in replacement for
\texttt{PAR}.

\subsection{Procedures and Recursion}

I did not implement procedures in my compiler, but if I had they would likely
have not allowed recursion. This is due to the way that processes are laid out
in memory by the compiler. However, with the distributed model, it would be
possible to implement recursion by spawning recursive calls in new instances.
To make this more efficient, it would probably be sensible to have different
instructions for explicitly requesting the instances to be spawned on the same
machine.

\subsection{Load Balancing}

The manner by which instances are distributed between process servers in my
implementation is a simple round-robin approach. However, it would be entirely
plausible for the process master to periodically query the relative loads of
each worker, and to take into account different factors such as the amount of
free memory on each worker, and the relative speeds of each processor if the
workers are not running on a homogeneous collection of machines.
